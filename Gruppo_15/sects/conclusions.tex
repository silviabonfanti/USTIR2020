\section{Summary and Outlook}

The transition from petroleum-based resins to greener alternatives have already started, but there are still some concerns that need to be addressed. Most parts produced using bio-resins have inferior mechanical and thermal properties when compared with petroleum-based resins or with other additive manufacture methods. Part of the problem is the lack of bio photosensitive resins with short acrylate monomer. Another problem is the small variety of materials used for photopolymerization. Plant-based resins have just started to be used, but more research needs to be made.

Even though some resins (plant-based, PEGDA, PEGMEMA, poly(propylene fumarate), trimethylene carbonate) have shown some biocompatibility, the harsh nature of UV polymerization and the lack of biocompatible photo-initiators still present multiple challenges for much medical application developed by SL. Photocurable resins under visible light would allow better cellular growth and medical parts with lower toxicity.

Nevertheless, the applicability of stereolithography will continue to grow as new bio-resins are being developed. Moreover, there is a variety of nanomaterials yet to be incorporated and tested, which can improve thermo-mechanical performances and add other properties.