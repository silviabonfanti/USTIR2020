\section{Introduction}

Additive manufacturing (AM), also known as solid free-form fabrication or layer-based manufacture, is a method of creating physical objects from digital models by depositing thin layers of materials in a “layer-by-layer” process.Its different from traditional subtractive manufacturing (the process consists of removing material from a bulk part) since it can work with different materials at the same time and does not waste many resources. Additive manufacturing also allowed designing and prototyping be accomplished fasters (in few hours or days), the creation of customer-specific designs and to rapid scale on-demand, which is revolutionizing the manufacturing sector entirely.

According to the ISO/ASTM 52900, it is possible to divide the additive manufacturing techniques into six categories (Figure 1), where each technology was developed to process different forms and types of materials.

\begin{figure}[h]
	\centering
	\includegraphics[width=0.7\linewidth]{"figs/Add Manuf Tech"}
	\caption{Examples of Add Manufacture}
	\label{fig:add-manuf-tech}
\end{figure}
