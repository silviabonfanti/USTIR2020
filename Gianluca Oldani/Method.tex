\documentclass[conference]{IEEEtran}
\IEEEoverridecommandlockouts
% The preceding line is only needed to identify funding in the first footnote. If that is unneeded, please comment it out.
\usepackage{cite}
\usepackage{amsmath,amssymb,amsfonts}
\usepackage{algorithmic}
\usepackage{graphicx}
\usepackage{textcomp}
\usepackage{xcolor}
\def\BibTeX{{\rm B\kern-.05em{\sc i\kern-.025em b}\kern-.08em
    T\kern-.1667em\lower.7ex\hbox{E}\kern-.125emX}}
\begin{document}

\title{Conference Paper Title*\\
{\footnotesize \textsuperscript{*}Note: Sub-titles are not captured in Xplore and
should not be used}
\thanks{Identify applicable funding agency here. If none, delete this.}
}

\author{\IEEEauthorblockN{1\textsuperscript{st} Given Name Surname}
\IEEEauthorblockA{\textit{dept. name of organization (of Aff.)} \\
\textit{name of organization (of Aff.)}\\
City, Country \\
email address or ORCID}
\and
\IEEEauthorblockN{2\textsuperscript{nd} Given Name Surname}
\IEEEauthorblockA{\textit{dept. name of organization (of Aff.)} \\
\textit{name of organization (of Aff.)}\\
City, Country \\
email address or ORCID}
\and
\IEEEauthorblockN{3\textsuperscript{rd} Given Name Surname}
\IEEEauthorblockA{\textit{dept. name of organization (of Aff.)} \\
\textit{name of organization (of Aff.)}\\
City, Country \\
email address or ORCID}
\and
The present article contributes to previous research by summarizing and critically investigating the research taken on the topic of startup valuation, which is one way to contribute conceptually to the literature (Gummerus, 2013). This literature review aims to comprehensively synthesize research that bears on a particular question using organized, transparent, and replicable procedures at each step in the process (Littell, 2008) to identify, in our case, the contributions in the field of the startup company valuation.
According to Fink (2019), we followed four steps for this systematic review. In the first step, we selected the research question, databases, websites and appropriate research terms. The research question was rather broad: ―What are better methods to assess startup value today?‖. To search the literature, we chose numerous variants of keywords focused on startup valuation. The databases searched were those provided by major publishers and by library services. We also used Google Scholar to identify other relevant material, such as conference proceedings and similar publications. The second step is the application of practical screening criteria. We considered refereed academic papers, books, research reports, conference proceedings and practitioner-oriented contributions from professional journals written in English without a time restriction. The third step is the application of methodological screening criteria. We considered, for each source, bibliographic data (authors, year, title, etc.), the background of the publication (is it theoretical/conceptual, empirical or practical oriented source?), and the focus and content of the publication. The last step consists of expositions of our analysis results.